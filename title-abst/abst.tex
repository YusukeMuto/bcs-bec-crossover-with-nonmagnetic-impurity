\documentclass[12pt]{jsbook}
\pagestyle{empty}
\begin{document}

\chapter*{論文要旨}
本修士論文では、冷却フェルミ原子気体のBCS (Bardeen-Cooper-Schrieffer)-BEC (Bose-Einstein condensation)領域における非磁性不純物効果を理論的に研究する。近年、この系に対し、不純物散乱効果を実験的に導入することが可能となっている。ここでは、引力相互作用する2成分フェルミ原子気体中に有限濃度の非磁性不純物がある場合を考え、不純物散乱の効果を自己無撞着$T$行列理論を用い研究する。絶対零度の超流動状態をBCS-Leggett理論の枠組みで扱い、超流動状態がどう不純物散乱により抑制されるかを、BCS-BECクロスオーバー全域で明らかにする。また、有限温度の正常相に対する強結合$T$行列理論を非磁性不純物が存在する場合に拡張し、ユニタリ極限における超流動転移温度$T_{\rm c}$への不純物効果、および、この領域の$T_{\rm c}$近傍に現れる擬ギャップ現象への影響を研究する。
\par
第 1 章では、極低温フェルミ原子気体の研究分野を概観し、金属超伝導の分野で議論されてきた不純物効果、および、冷却原子気体に不純物を導入する実験方法について説明する。その後、研究目的を述べる。
\par
第 2 章では、本論文で用いる理論の定式化を行う。先ず、非磁性不純物散乱に対する自己無撞着$T$行列理論をファインマンダイヤグラムを用い説明し、次に、それを絶対零度の超流動状態に対するBCS-BECクロスオーバー理論の1つであるBCS-Leggett理論に取り込む方法を述べる。更に、同様の拡張を、BCS-BECクロスオーバー領域における超流動転移温度に対する強結合$T$行列理論に対し行う。
\par
第 3 章では、先ず、絶対零度の超流動状態を考え、超流動秩序パラメータ$\Delta$に対する非磁性不純物効果を議論する。不純物散乱強度が強くフェルミ原子が不純物に束縛されるようになると、$\Delta$は小さくなり、特に、全てのフェルミ原子が不純物に束縛され不純物バンドが完全に占有される状況では、超流動状態が消失することを示す。また、非磁性不純物散乱は$T_{\rm c}$にも影響を与えることを、ユニタリ極限において明らかにする。この領域では、$T_{\rm c}$近傍で擬ギャップ現象が起こることが理論的に指摘されているが、非磁性不純物がこの多体現象にどう影響するか、についても議論する。
\par
第 4 章では本修士論文の内容をまとめ、今後の課題を述べる。
\par
\chapter*{Thesis Abstract}
\section*{Nonmagnetic impurity effects in the BCS-BEC crossover regime of an ultracold Fermi gas}
\par
In this master thesis, I theoretically investigate nonmagnetic effects on an ultracold Fermi gas in the BCS (Bardeen-Cooper-Schrieffer)-BEC (Bose-Einstein condensation) region. Recently, it has become possible to experimentally introduce non-magnetic impurities to the system in cold atom physics. In this thesis, I consider a two-component interacting Fermi gas involving nonmagnetic impurities. I deal with the superfluid state at $T=0$ within the framework of the BCS-Leggett strong-coupling theory, to clarify how the superfluid phase is affected by nonmagnetic impurity scatterings, in the whole BCS-BEC crossover region. I also study nonmagnetic impurity effects on the superfluid phase transition temperature $T_{\rm c}$, as well as the so-called pseudogap phenomenon appearing near $T_{\rm c}$, in the unitarity limit.
\par
In Chap. 1, after an introduction about cold Fermi gas physics, I briefly review impurity effects on metallic superconductors. I also explain how to experimentally introduce nonmagnetic impurities to cold atom gases. I then present the purpose of this thesis.
\par
I explain my formulation in Chap. 2. I first present a self-consistent $T$-matrix theory in terms of impurity scatterings, by using Feynman diagrams. Next, I explain how to incorporate this scattering theory into the BCS-Leggett strong-coupling theory, which is valid for the superfluid phase at $T=0$ in the BCS-BEC crossover region. I also present a similar extension for a strong-coupling $T$-matrix theory, to deal with $T_{\rm c}$ in the crossover region.
\par
In Chap. 3, I discuss the superfluid order parameter $\Delta$ and nonmagnetic impurity effects at $T=0$. When the impurity scattering is so strong that some Fermi atoms are bound by impurities, $\Delta$ is shown to be suppressed. In particular, the superfluid phase disappears in the extreme case when all the Fermi atoms form impurity-atom bound states and the impurity band is completely occupied by Fermi atoms. At the unitarity, I also clarify that impurity scatterings also affect $T_{\rm c}$, as well as the pseudogap phenomenon.
\par
I summarize this thesis in Chap. 4, where I also give some future problems.
\end{document}
