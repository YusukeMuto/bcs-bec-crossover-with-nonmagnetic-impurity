\chapter*{謝辞}

本研究は、著者が慶應義塾大学大学院理工学研究科修士課程在学中に、
同大学理工学部大橋洋士教授の指導のもとに行ったものである。
\\

本研究を遂行するにあたり、ご指導及びご協力いただきました全ての皆様に心より感謝申し上げます。
\\

特に、様々なご指導を頂きました指導教員の大橋洋士教授に深謝いたします。大橋洋士教授には学部4年の卒業研究から研究活動における指導をしていただきました。本論文をまとめることができましたのは、ひとえに大橋洋士教授のご指導の賜物です。私が寄り道をしていた時にも、大橋洋士教授が辛抱強く指針を示してくださっていたことに心から感謝しています。また大橋洋士教授には研究面のみならず、生活面においても多くのことをご教授いただきました。
\\

渡邉紳一准教授、檜垣徹太郎専任講師には本論文の副査を引き受けていただきました。渡邉紳一准教授には、分かりづらい箇所から、本質的な疑問まで、多くの指摘をいただきました。檜垣徹太郎専任講師には、本修士論文で用いた枠組みなど、基礎的でかつ重要な部分に関して改めて考える機会をいただきました。本論文の執筆にあたり貴重なご意見を頂いたことに、心より感謝申し上げます。
\\

物理学科の教職員の方々、特に、白濱圭也教授、江藤幹雄教授、齊藤圭司准教授、山本直希専任講師には、授業や日常的な会話、あるいは研究に関する議論を通して、様々な刺激を頂いたことに感謝申し上げます。
\\

計算物理学実習、プログラミング実習、物理学演習第3 の TA として私を指導してくださった山内淳准教授、光武亜代理専任講師、檜垣徹太郎専任講師、また経済学部などを対象とした UNIX や C 言語、C++ 言語の授業において、様々なことを TA として学ばせていただいた丸山文綱先生、恩田憲一先生には、至らない点の多い私を最後まで指導してくださり、研究に活きるような知識も多く学ばせていただいたことに感謝しています。
\\

理論研究室大橋グループの猪谷太輔助教、PD 研究員の花井亮氏、博士課程の田島裕之氏、山口辰威氏、Digvijay Kharga 氏、松本杜青氏、Pieter van Wyk 氏、Soumita Mondal 氏、鏡原大地氏、前学期まで前期博士課程に在籍されていた太田幹氏には、様々なご協力賜り、厚く御礼申し上げます。

猪谷太輔助教にはいつも研究に関するアドバイスをいただき、問題の解決に導いていただきました。猪谷助教が考えた通りの結果が出ることも多く、驚かされながら多くのことを学ばせていただきました。お世話になったことに感謝しています。

花井亮氏には研究生活全般の相談に乗っていただきました。本研究の定式化に関しても様々なことを教えていただきました。花井亮氏がいなければ諦めてしまっていたことも多くあったと思います。いつも気にかけて頂き、本当にありがとうございました。感謝の気持ちを述べても述べ足りないです。

鏡原大地氏には、本研究における BEC 側での不純物効果等に関する議論をしていただき、お世話になりました。また研究室の向かいの席でもあり、いつも親切にしてくださったことに感謝しています。

田島裕之氏、Digvijay Kharga 氏、松本杜青氏、Pieter van Wyk 氏には、研究に関して多くの有益な助言をしていただきました。

山口辰威氏、Soumita Mondal 氏、太田幹氏には、大橋グループのゼミにおいて大変お世話になりました。


いつでも優しく、厳しく、行き届いた指導をしてくださる大橋グループで研究できたことは、私の人生において大きな糧となりました。繰り返しとなりますが、大橋グループの皆様に心より感謝申し上げます。
\\

理論研究室の皆様、特に、西田有延氏、川久保龍一郎氏、横倉諒氏、玉木修二氏、および、同期で学部時代から自主ゼミなどを通して切磋琢磨してくれた岩崎諄氏、川島聡氏、木村綾斗氏、曽我部紀之氏、花里太郎氏、松永拓氏、若村浩明氏、また数物セミナーなどの集まりで一緒に学ぶことができた阿部慶彦氏、ピーダーセン珠杏氏、西村健太郎氏、工藤勇氏、平澤尚之氏、物理学科の学部時代に同期であった真崎悠氏、大島駿太郎氏、野口千尋氏、磯貝基氏、神田大輝氏、中村友貴氏、岡野元基氏、森将氏のおかげで、物理の理解が深まっただけでなく、物理学科での生活は非常に充実したものになりました。ありがとうございました。中でも理論研究室の同期の存在は非常に大きかったです。
\\

\clearpage
物理学科事務の米内山裕子氏には、ゼミ室の予約から物理学会関係の書類、その他事務手続きにおいて、円滑に研究生活を送る上で、多くのことをお世話になりました。また事務の方が気さくに話しかけてくださることは、少なからず研究生活の充実につながっていたと思います。ありがとうございました。
\\

ピアノクラブの皆様、修士課程を無事に終えることができたのは、いつでも研究に関係なく、心を許せる仲の良い友人として、皆様がいてくれたからだと思っています。ありがとうございました。
\\

幼い頃からピアノの指導をしてくださり、私にとって第二の母と言っても過言ではない有山令子氏にも心よりの感謝を申し上げます。有山令子氏のもとでピアノを習っていたからこそ、今でもピアノを続けることができ、研究が煮詰まった時などにピアノを弾くことが最大のリラックス法になっているのだと思います。本当にありがとうございます。
\\

最後に、大学院で学ぶことを含め、私が決めたことをいつでも応援してくださり、不自由のない生活をさせてくださった家族に、心より感謝申し上げます。
