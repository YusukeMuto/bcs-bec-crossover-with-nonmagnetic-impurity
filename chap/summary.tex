\chapter{まとめ}

本修士論文では、極低温フェルミ原子気体に対する非磁性不純物効果を、絶対零度の BCS-BEC クロスオーバー領域における超流動相、およびユニタリ極限における超流動転移温度 $\tc$ で理論的に研究した。理論の枠組みとしては、不純物に対しては自己無撞着 $T$ 行列理論で扱い、また分布平均を取ることで理論の並進対称性を回復させた。絶対零度の超流動相に対しては BCS-Leggett の強結合理論、$\tc$ に対しては $T$ 行列理論を用いた。

金属超伝導の分野ではフェルミ原子ガス超流動のような“等方的 $s$ 波超流動”(超流動秩序パラメータ $\del$ が波数依存性を持たないフェルミ超流動)に対応する $s$ 波超伝導では、超伝導の $\tc$ や $\del$、および熱力学性質が非磁性不純物散乱の影響を受けないというアンダーソンの定理がよく知られている。しかし、そこでは電子間引力相互作用が及ぶ範囲がフェルミ面のごく近傍に限られ、それにより電子-正孔対称性がよい近似で成立するといった仮定が用いられる。本研究は相互作用が可変で、かつ、物理的カットオフが存在せず、全相互作用領域に及ぶフェルミ原子気体に対しては、この定理が必ずしも成立せず、不純物散乱が強く、フェルミ原子が不純物と束縛状態を形成してしまうような状況では、超流動状態が消失してしまうことさえ起こりうることを具体的な計算により示した。

非磁性不純物散乱により超流動秩序パラメータ $\del$ や超流動転移温度 $\tc$ が抑制される原因はこの散乱により 1 粒子状態密度が常流動相の時点で影響を受けることによるものである。すなわち、不純物とフェルミ原子間の相互作用(散乱ポテンシャル)が強くフェルミ原子が全て不純物に束縛され、かつそれらの“伝導性”が完全に抑制されてしまうと、系は超流動状態にはなれない。本研究では接触型の散乱ポテンシャルを有する不純物(これは 1 不純物あたり擬スピン $\spin=\uar, \dar$ の計 2 粒子を束縛することができる)に対し、不純物濃度がフェルミ原子ガスの濃度の半分の場合に実際超流動状態の消失が起こることを明らかにした。これは、自由フェルミガスのバンドの下に形成された“不純物バンド”が完全に占有され、ある種のバンド絶縁体になった状況で実現する。一方、不純物濃度がこの条件より低い場合は不純物に束縛されなかったフェルミ原子により、超流動状態は保持される。また不純物濃度が高く全てのフェルミ原子を束縛した後もまた原子を束縛できる不純物が残っている状況では、不純物バンドが完全に満たされないため絶縁体的にはならず超流動状態は残る。ただしこの場合、実現する $\del$ や $\tc$ は非常に小さくなり、今回の数値計算では不純物散乱が非常に強い状況まで超流動状態が残り続けるかは完全には検証できなかった。これに関しては今後の課題である。

以上の非磁性不純物効果は散乱によりクーパー対が破壊されることで超伝導状態が抑制される磁性不純物効果とは異なり、フェルミ原子間引力相互作用を考えない範囲での不純物散乱の影響を受けた 1 粒子状態密度の構造に起因するものである。ただし、こうした状態密度の効果に加え、本研究では、不純物とフェルミ原子による束縛エネルギーがクーパー対の結合エネルギーに近い場合、前者の方がエネルギー的に得になるとクーパー対を破壊し、不純物との束縛状態を形成することで超流動状態が抑制されることを理論的に示した。これは非磁性不純物であっても対破壊効果を与えることがあることを示すものである。

フェルミ原子気体の BCS-BEC クロスオーバー領域では、強い引力相互作用により $\tc$ 以上でも 1 粒子状態密度に超流動状態密度に似た擬ギャップと呼ばれる窪み構造が生じることが知られているが、本研究でもユニタリ極限の $T_c$ において、この現象に対する非磁性不純物効果を研究し、擬ギャップ構造が不純物散乱の影響をどう受けるか、を議論した。

本論文では、$\tc$ に対してはユニタリ極限のみを考えたが、これを全 BCS-BEC クロスオーバー領域で行い、従来のフェルミ原子気体の「温度-相互作用相図」に「不純物濃度」、「不純物散乱強度」の軸を加えて相図を拡張することは今後の課題である。またユニタリ領域の低不純物濃度において $\tc$ が $\bskfi=1$ のあたりで一度大きく減少する結果を得たが、その原因を明らかにすることも今後必要な研究である。これまでフェルミ原子ガス超流動の研究はクリーンな系に対し行われてきたが、近年不純物効果を導入する実験が可能となり、BCS-BEC クロスオーバー現象に対する不純物効果が研究できるようになった。本研究はアンダーソンの定理の枠を越え、この問題を研究する上で非常に役立つものと期待される。

